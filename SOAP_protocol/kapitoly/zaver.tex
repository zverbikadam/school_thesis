\chapter*{Záver}
\addcontentsline{toc}{chapter}{Záver}

Ako sme sa dozvedeli, protokol SOAP má mnoho výhod. Je nezávislí na platforme a programovacom jazyku, ako formát používa XML, ktorý je veľmi rozšírený a má veľkú dostupnosť nástrojov ponúkaných ako \textit{opensource} alebo \textit{freeware}. Taktiež je syntax XML jednoducho čitateľná pre človeka. Protokol SOAP je považovaný za zabezepečenú webovú služby vďaka implementácií nazvanej ako \textit{WS Security} \cite{Security}. V dnešnej dobe sa používa hlavne vo finančných a bussiness aplikáciach. Finančné spoločnosti preferujú SOAP hlavne preto, že \textit{WS-Atomic Transactions} má schopnosť dvojfázového potvrdenia transakcií. Dvojfázové potvrdenie je akceptovaný spôsob spracovania mnohých typov finančných aktualizácií, pretože táto funkcia umožňuje databázam návrat do stavu pred transakciou, pokiaľ sa behom spracovania požiadavky stane nejaká chyba \cite{Trends}.

Síce má protokol SOAP mnoho výhod, má aj nevýhody. Umožňuje iba použitie formátu XML, a teda napríklad formát \textit{JSON}, alebo iné formáty nie sú povolené. Ďalšou jeho veľkou nevýhodou bola dlhá syntax, ktorú počítač musí zložito parsovať, čo stojí veľa procesorového času. Vzhľadom k tomu, že vývoj ide stále dopredu a takmer každým dňom výkon počítačov narastá, tento problém už nie je úplne aktuálny. Taktiež existuje aj binárne XML, ktoré je externou reprezentáciou hodnoty XML, ktorá môže byť použitá na výmenu XML dát medzi klientom a serverom. Binárna reprezentácia poskytuje efektívne parsovanie, čo môže viesť k zlepšeniu výkonu pre výmenu XML dát \cite{Binary}.

Aj napriek tomu, že SOAP je \textit{"prarodič"} ostatných webových služieb, stále má využitie a vďaka SOAP 1.2 tomu bude tak aj v budúcnosti. SOAP sa najlepšie hodí, ak aplikácia vyžaduje zaručenú úroveň spoľahlivosti a bezpečnosti, SOAP 1.2 ponúka štandarty na tento typ operácií (napr. WSRM - WS Reliable Messaging). Ďalej, ak sa potrebujú obe strany, tj. poksytovateľ (provider) a spotrebiteľ (consumer), dohodnúť na výmennom formáte, SOAP 1.2 poskytuje prísne špecifikácie a schémy pre tento typ interakcie. Taktiež v prípade stavových operácií - ak aplikácia potrebuje kontextové informácie a správu o konverzačnom stave, potom SOAP 1.2 má dodatočnú špecifikáciu v štruktúre WS na podporu vecí ako bezpečnosť, transakcie, koordinácia atď \cite{future}.