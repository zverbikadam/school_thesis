\chapter*{Úvod}
\addcontentsline{toc}{chapter}{Úvod}

Pre väčšinu ľudí v dnešnej dobe predstavuje internet nedeliteľnú časť ich života. Vďaka nemu majú prístup k správam a informáciam z celého sveta behom pár kliknutí, môžu si pozerať obľúbené filmy, či seriály kedykoľvek chcú, alebo môžu komunikovať s ďaľšími ľuďmi na opačnom konci sveta. Aby bolo toto možné, musia nejakým spôsobom komunikovať aj jednotlivé zariadenia, pomocou ktorých sa ľudia na internet pripájajú. V súčasnosti práve požiadavky na vzájomnú komunikáciu úplne odlišných systémov naberajú na dôležitosti. Samotná komunikácia ako nástroj výmeny informácií nie je dostatočná a musí spĺňať alebo podporovať kľučové vlastnosti súčasných systémov akými je škálovateľnosť, bezpečnosť, stabilita, tolerancia k poruchám a vysoká dostupnosť.

Odhliadnuc od hardvérových prvkov ako sú switche, routre, a podobne, práve tu prichádzajú na rad pojmy ako webové služby a protokol SOAP. Okrem protokolu SOAP samozrejme existuje množstvo iných spôsobov riešenia tohto problému. Niektoré staršie sú napríklad CORBA, DCOM alebo RMI. Ďalej to môžu byť nástroje založené na moderných trendoch a princípoch ako napríklad JSON a REST API (alebo RESTful API).

